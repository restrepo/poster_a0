\documentclass{beamer}
%%%%%%%%%%%%%%%%%%%%%%%%%%%%%
\usepackage{talk}


\widescreen


\title{Title}
\subtitle{Sibtitle  }


%\subtitle
%{Reconstruction of the neutrino mass matrix} % (optional)

\author{Diego Restrepo}
% - Use the \inst{?} command only if the authors have different
%   affiliation.
\institute{\affiliation{Instituto de F\'\i sica\\
Universidad de Antioquia\\
Phenomenology Group}%
{http://gfif.udea.edu.co}%
{Image title}%
{imageicon}%
{Image credits}
\arxiv{arXiv:NNNN.NNNNN (Jrnl}
\collaborators{NN}
}
% - Use the \inst command only if there are several affiliations.
% - Keep it simple, no one is interested in your street address.

\date{\tiny November 15, 2016 } % (optional) \today
%{\includegraphics[scale=0.3]{udea}}
\titlegraphic{\hfill\includegraphics[height=1.5cm]{udea}}


\begin{document}


%===============
\begin{comentar}
%===============  
%=============
\end{comentar}
%=============

\maketitle

\begin{frame}
  \frametitle{Table of Contents}
% %\small
% %\vspace{-0.5cm}
   \setbeamertemplate{section in toc}[sections numbered]
   \tableofcontents[hideallsubsections]
   \tableofcontents[hideallsubsections]
\end{frame}

\section{First section}

\begin{frame}
  \frametitle{First title}

\end{frame}

\end{document}





TEMPLATES

1. Background image slide
%%%%%%%%%%%%%%%%%%%%%%%%%%%%%%
{
\usebackgroundtemplate{\includegraphics[width=\paperwidth]{file}}
\setbeamertemplate{blocks}[rounded][shadow=false]
\setbeamercovered{invisible}
\begin{frame}[plain]
\end{frame}
}
%%%%%%%%%%%%%%%%%%%%%%%%%%%%%%

2. Two columns
\begin{columns}
  \begin{column}{0.48\textwidth}
    
  \end{column}
  \begin{column}{0.48\textwidth}
    
  \end{column}
\end{columns}




\begin{columns}
  \column{.48\textwidth}
  \begin{block}<2->{}
  \end{block}
  \column{.48\textwidth}
  \begin{block}<2->{}
  \end{block}
\end{columns}

%%%%%%%%%%%%%%%%%%%%%
{
\usebackgroundtemplate{\includegraphics[width=\paperwidth]{file}}
\setbeamertemplate{blocks}[rounded][shadow=false]
\setbeamercovered{invisible}
\begin{frame}[plain]
  \begin{block}{}
    Name
  \end{block}
\end{frame}
}
%%%%
{
\usebackgroundtemplate{\includegraphics[width=\paperwidth]{file}}
\setbeamertemplate{blocks}[rounded][shadow=false]
\setbeamercovered{invisible}
\begin{frame}[plain]
\end{frame}
%%%%%%%%%%%%%%%%%%
}


%Trick to put stuff in specfic places of an slide:
\begin{frame}
\begin{picture}(320,250)
\put(-25,190){D.R. \emph{et al}: arXiv:1006.5075 [PRD]\qquad\qquad arXiv:1206.3605 [PRD]}
\put(-37,20){\includegraphics[scale=0.35]{atmcorrelationm}}
\put(143,20){\includegraphics[scale=0.35]{LoverWlZnu_Dm23_500_500_randomm}}
\put(0,160){Only depend in \alert{$\Lambda_i$}}
\end{picture}
\end{frame}
%%%%%
Background like
%%%%%%%%%%%%%%%%%%%%%%%%
\begin{frame}[plain]
\begin{picture}(320,250)
\only<1>{\put(-30,-15){\includegraphics[width=\paperwidth]{brpv1}}}%
\only<2>{\put(-30,-15){\includegraphics[width=\paperwidth]{brpv2}}}%    
\only<2>{\put(-30,-15){\includegraphics[width=\paperwidth]{brpv3}}}%    
\end{picture}
\end{frame}


%%%
Post it

\setbeamercolor{postit}{fg=black,bg=yellow}
\begin{beamercolorbox}[sep=1em,wd=5cm]{postit}
Place me somewhere!
\end{beamercolorbox}

%%%combine with textblock
\begin{textblock*}{297mm}(0mm,0mm)%
\begin{beamercolorbox}[sep=0.1em,wd=4cm,center,rounded=true,shadow=true]{cite}
\scriptsize Akhmedov, hep-ph/0001264
\end{beamercolorbox}
\end{textblock*}

%%More colorboxes
\setlength{\fboxrule}{3 pt}
\fcolorbox{red}{yellow}{caja de fondo
amarillo y contorno rojo}\\
\setlength{\fboxsep}{5pt}
\fcolorbox{red}{yellow}{caja de fondo
amarillo y contorno rojo}

\only<11>{\put(-20,80){
\begin{minipage}[t]{1.0\linewidth}
%\metroset{block=fill}
\begin{block}{SM particles}
     Gauge invariance+Lorentz Invariance$\to$Lagrangian (interactions)   
\end{block}
    \end{minipage}
}}


\begin{frame}
\begin{picture}(320,250)
\only<1->{\put(0,100){
  \begin{overpic}[scale=0.4,grid
            ]{table1}
\put(40,38){\tikz \draw[red,thick,rounded corners] (0,0) rectangle (2,0.5);}
  \end{overpic}
}
}
%%%circle
\end{picture}
\end{frame}
